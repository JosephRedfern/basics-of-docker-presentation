\documentclass[10pt]{beamer}

\usetheme{metropolis}
\usepackage{appendixnumberbeamer}

\usepackage{booktabs}
\usepackage[scale=2]{ccicons}
\usepackage{minted}
\usepackage{xcolor}
\usepackage[utf8]{inputenc}
\usemintedstyle{manni}

\setminted{fontsize=\small,baselinestretch=1}%,bgcolor=LightGray}

\usepackage{pgfplots}
\usepgfplotslibrary{dateplot}

\usepackage{xspace}
\newcommand{\themename}{\textbf{\textsc{metropolis}}\xspace}


\title{Packaging Software with Docker}
\date{\today}
\author{Joseph Redfern}

\begin{document}

\maketitle

\begin{frame}[fragile]{The Problem}
	\vfill Software is fast moving. 

	\vfill Scripts or tools that worked 2--3 years ago (or less) may not work now due to changes to the system running the software.

	\vfill Software X may require library version A, software Y may require library version B.

	\vfill Breakages can be called by operating system upgrades, updates to dependencies.

\end{frame}

\begin{frame}[fragile]{The Problem (cont.)}
	\vfill Some ecosystems try to solve this problem with virtual environments:

	\vfill \begin{itemize}
	\item \verb|pip| and \verb|requirements.txt| 
	\item \verb|anaconda| and \verb|environment.yml|
	\item Tools like \verb|npm|/\verb|nvm| for node.js packages/version, \verb|gem|/\verb|rvm| for ruby packages/versions
	\end{itemize}

	\vfill These solutions solve some (but not all) of these issues.

	System library requirements can be issue, even down to things like version of \verb|glibc|.

	Often require the end user be familiar with the ecosystem, not necessarily the case.
\end{frame}

\begin{frame}[fragile]{What is docker?}
	\vfill Builds on standard features from Linux kernel, like \verb|cgroups|, \verb|netfilter|, \verb|namespaces|.

	\vfill Provides a set of tools that allows developers to package and distribute software as stand-alone containers.

	\vfill A container is \textit{like} a lightweight virtual machine -- boot time can be in milliseconds.

	\vfill \textbf{However}, shares Kernel of host machine (unlike more traditional virtualisation), but not userspace.

\end{frame}

\begin{frame}[fragile]{What is docker? (cont.)}
	
	\vfill A container has its own filesystem, but you can map directories from the host to the container if desired.

	\vfill Registry of base ``images'' on which to build -- these images typically the userspace (libraries/system utilities) of existing operating systems, or preconfigured environments for existing software packages.

	\vfill Docker (the company) provide an image registry (docker hub). University's GitLab install also features Docker registry.
\end{frame}

\begin{frame}[fragile]{Example usage}

	\vfill To distribute software through Docker, we need to create an image.
	\vfill To create an image, we need to write a \verb|Dockerfile|.
	\vfill The \verb|Dockerfile| contains the commands needed to assemble our image.
	\vfill Includes reference to parent container, defines working directory, copies files and resources from build machine to container, runs commands in the container before being snapshotted.
	\vfill Also defines commands to be run when container starts.
\end{frame}

\begin{frame}[fragile]{Example Dockerfile}
	\inputminted{Dockerfile}{example/Dockerfile}
\end{frame}

\begin{frame}[fragile]{Building an image}
	\vfill Once we've defined our \verb|Dockerfile|, we need to build the image.
	\vfill Use the command: \verb|docker build . -t name:version|.
	\vfill The \verb|-t| arg lets us tag the image with a name \& version -- for instance, \verb|-t awesomesoft:1.3.37|.
\end{frame}

\begin{frame}[fragile]{Demo}

\vfill <fragile demo goes here>\vfill

\end{frame}

\begin{frame}[fragile]{Parting comments}
\vfill Docker makes it easy to distribute functional software, minimising barrier to entry. Helpful for reproducible research.

\vfill Containers can be really lightweight. Minimal performance hit, very little overhead. 

\vfill \textbf{Some} security benefits -- can limit access to filesystem, restrict network access, limit memory/CPU usage. (not without pitfalls) 

\vfill Can use cuda within docker using \verb|nvidia-docker|.

\vfill Possible to pass USB devices to docker containers with \verb|--device| flag.

\vfill Compatible with Windows and macOS (through virtualisation).

\end{frame}

\end{document}
